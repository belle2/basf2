\subsection{Data flow}
\secwriter{P. Urquijo}

\subsection{Modular analysis approach}
\secwriter{P. Urquijo, A. Zupanc}

A typical analysis can be separated into well defined analysis actions. For example, in a measurement of time-dependent $CP$ violation
in $B^0\to \phi K^0_S$ decays we need to execute the following actions:
\begin{itemize}
\item create list of charged kaon candidates
\item create list of $\phi$ candidates by making combinations of two oppositely charged kaons
\item create list of $K^0_S$ candidates
\item create list of $B^0$ candidates by making combinations of $\phi$ and $K^0_S$ candidates
\item calculate continuum suppression variables
\item determine the flavor of $B^0$ candidates
\item determine the decay vertex of $B^0$ candidates
\item determine the decay vertex of the other $B$ meson in the event
\item write out all relevant info to ntuple for offline analysis
\end{itemize}
A measurement of time-dependent $CP$ violation in $B^0$ decays to a different final state, e.g. $J/\psi K^0_S$, $D^+D^-$ or 
$K^{\ast0}\gamma$, would consist of conceptually identical analysis actions. The only difference is in the intermediate and final state
particles. Therefore, commonly used analysis tools need to be prepared in order to enable efficient and accurate data analysis. 

Plans is to develop for each analysis action an analysis module so that the whole analysis sequence given above can be executed
within a python script.