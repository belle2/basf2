\documentclass[a4paper,11pt]{article}
\pdfoutput=1

\usepackage{jheppub} % for details on the use of the package, please
                     % see the JHEP-author-manual

\usepackage{hyperref}
\usepackage{fancyvrb,color}
\usepackage{colortbl}
\usepackage{xspace}
\usepackage[T1]{fontenc} % if needed


\subheader{
Draft-Version-0.1 (Latex run \today)\\
}

\title{\boldmath Belle 2 Analysis Software Design Documentation}

% if you make contribution to this document please add your name and affiliation to the list
\author[a]{C.~Oswald}
\author[b]{C.~Pulvermacher}
\author[c]{M.~Stari\v{c}}
\author[a]{P.~Urquijo}
\author[b]{A.~Zupanc}
\affiliation[a]{University of Bonn, Bonn, Germany}
\affiliation[b]{Karlsruhe Insitute Of Technology, Karlsruhe, Germany}
\affiliation[c]{Jo\v{z}ef Stefan Institute, Ljubljana, Slovenia}

\abstract{
We summarize the Belle 2 analysis sofware design decisions and discuss usecases.
}

\newcommand{\mdst}{{\tt mDST}\xspace}
\newcommand{\mudst}{{\tt $\mu$DST}\xspace}

\newcommand{\dStore}{{\tt DataStore}\xspace}
\newcommand{\relation}{{\tt Relation}\xspace}

\newcommand{\track}{{\tt Track}\xspace}

% DATA Model Class names
\newcommand{\particle}{{\tt Particle}\xspace}
\newcommand{\mcParticle}{{\tt MCParticle}\xspace}
\newcommand{\vertex}{{\tt Vertex}\xspace}
\newcommand{\pList}{{\tt ParticleList}\xspace}
\newcommand{\pidLikelihood}{{\tt PIDLikelihood}\xspace}

% Module names
\newcommand{\pLoader}{{\tt ParticleLoader}\xspace}
\newcommand{\pSelector}{{\tt ParticleSelector}\xspace}
\newcommand{\pCombiner}{{\tt ParticleCombiner}\xspace}
\newcommand{\pVertexFitter}{{\tt ParticleVertexFitter}\xspace}
\newcommand{\ntMaker}{{\tt NtupleMaker}\xspace}

\newcommand{\plMerger}{{\tt ParticleListMerger}\xspace}

% commands
\newcommand{\secwriter}[1]{{\em Contribution from:}  #1\newline}


\begin{document} 
\maketitle
\flushbottom

\paragraph{What is software design document?}

{\it A software design document is a written description of a software product, that a software designer 
writes in order to give a software development team an overall guidance of the architecture of the software 
project. An software design document usually accompanies an architecture diagram with pointers to detailed 
feature specifications of smaller pieces of the design. Practically, a design document is required to coordinate 
a large team under a single vision. A design document needs to be a stable reference, outlining all parts of 
the software and how they will work. The document is commanded to give a fairly complete description, while 
maintaining a high-level view of the software.} (copied from wikipedia)

\section{BASF2 overview}
\secwriter{C.~Pulvermacher}

{\it The key elements and features (existing and planned) of BASF2 will be discussed, like \dStore, {\tt RelationsObject}, BASF2 relations.}

\section{Introduction to analysis software}
\subsection{Data flow}
\secwriter{P. Urquijo}

\subsection{Modular analysis approach}
\secwriter{P. Urquijo, A. Zupanc}

A typical analysis can be separated into well defined analysis actions. For example, in a measurement of time-dependent $CP$ violation
in $B^0\to \phi K^0_S$ decays we need to execute the following actions:
\begin{itemize}
\item create list of charged kaon candidates
\item create list of $\phi$ candidates by making combinations of two oppositely charged kaons
\item create list of $K^0_S$ candidates
\item create list of $B^0$ candidates by making combinations of $\phi$ and $K^0_S$ candidates
\item calculate continuum suppression variables
\item determine the flavor of $B^0$ candidates
\item determine the decay vertex of $B^0$ candidates
\item determine the decay vertex of the other $B$ meson in the event
\item write out all relevant info to ntuple for offline analysis
\end{itemize}
A measurement of time-dependent $CP$ violation in $B^0$ decays to a different final state, e.g. $J/\psi K^0_S$, $D^+D^-$ or 
$K^{\ast0}\gamma$, would consist of conceptually identical analysis actions. The only difference is in the intermediate and final state
particles. Therefore, commonly used analysis tools need to be prepared in order to enable efficient and accurate data analysis. 

Plans is to develop for each analysis action an analysis module so that the whole analysis sequence given above can be executed
within a python script.

\section{Data model}

General guidelines:
\begin{itemize}
 \item follow coding conventions given in \cite{basf2:coding:convention}
 \item public data members are forbidden
 \item avoid {\tt TObject}-derived data members (e.g. {\tt sizeof(TVector3)=40, 3*sizeof(double)=24}). Accessors can still return 
  convenient objects like TVector3. 
 \item clearly indicate what are {\it transient} and {\it persistent} data members
\end{itemize}
    
\newpage
\subsection{\particle}
\secwriter{A. Zupanc}

\paragraph{General overview}

This class is a common representation of all particle types, e.g.:
\begin{itemize}
 \item final state particles (FS particles)
 \begin{itemize}
  \item charged kaons, pions, electrons, muons, and protons reconstructed as {\tt Track}
  \item photons reconstructed as {\tt ECLGamma}
  \item long lived neutral kaons reconstructed in KLM
 \end{itemize}
 \item composite particles
 \begin{itemize}
  \item neutral pion pre-reconstructed as {\tt ECLPi0} 
  \item vee particles (short lived neutral kaon, $\Lambda$ baryon, converted photon) pre-reconstructed as {\tt MdstVee}\footnote{Note: 
  {\tt MdstVee} or equivalent data-object does not exist yet.}
 \item particles reconstructed by the user (via combinations)
 \end{itemize}
\end{itemize}

Private members are limited to those which completely define the particle and that are common to all particle types. Information which
exists and is unique only to a specific type of particles should not be saved inside the \particle class. Such information should be 
saved inside a specially designed data object which can be created and filled with some analysis module. This data object can be
related to the \particle objects of a specific type using BASF2 \relation. An example of such additional information and corresponding
data object are particle identification likelihoods which are stored in \pidLikelihood class. 

Private members are divided into persistent and transient data members. The former are those which can't be (easily) reproduced and therefore need
to be saved to \mudst while the latter can be easily reproduced (recalculated) on-demand.

\subsubsection{Data members}

The main and the most important data members of \particle class are the physics (kinematic) quantities:
\begin{itemize}
 \item 4-momentum vector and the point in which the momentum is estimated 
 \item the momentum-position error matrix (including the correlations between momentum and position estimates)
\end{itemize}
which enable reconstruction of unstable particles and determination of their decay (and production) vertices, etc... All other data members
are included in principle to make analyses user friendly, faster, ...

\paragraph{Persistent members} 
\begin{itemize}
 \item {\color{blue}energy in LAB} \hfill{float}
 \item {\color{blue}momentum vector in LAB} \hfill{$3\times$float}
 \begin{itemize}
  \item the point in which the momentum vector is estimated is given in the position/decay vertex data member
 \end{itemize}
 \item {\color{blue}position/decay vertex} \hfill{$3\times$float}
 \begin{itemize}
  \item in the case of charged FS particles this is point-of-closest-approach (POCA), which is copied from the \track data object.
  \item in the case of neutrals ($\gamma$, $\pi^0$, or $K^0_L$) this is the point of origin which is assumed during the 
  reconstruction of these particles (e.g. point $(0,0,0)$ or IP). This point is  copied from the corresponding \mdst data object. 
  \item in the case of composite particles this is the decay vertex determined by some vertexing module and until then this point is (0,0,0)
 \end{itemize}
 \item {\color{blue}$7\times7$ error matrix} \hfill{$28\times$float}
 \begin{itemize}
  \item order of elements in the error matrix is: $p_x, p_y, p_z, E, x, y, z$
  \item in the case of (pre-reconstructed) FS particles the error matrix is filled in \particle constructor
  \item in the case of composite particles the error matrix is filled when a kinematic fit is performed (until then is empty)
  \item internally the error matrix is saved as one dimensional array with 28 elements 
 \end{itemize}
 \item {\color{blue}$\chi^2$ probability of the fit} \hfill{float}
 \begin{itemize}
  \item in case of charged FS particles a $\chi^2$ value of the track fit is stored and in case of all composite particles the 
  $\chi^2$ value of the kinematic fit is stored (e.g. mass-constrained, vertex, mass-constrained vertex fit)
  \item variable is initialized to -1
  \begin{itemize}
   \item $\chi^2<0$ indicates that the error matrix is not valid
   \item $\chi^2\geq 0$ indicates that the error matrix is valid
  \end{itemize}
 \end{itemize}
 \item {\color{blue}number of degrees of freedom of the fit} \hfill{unsigned}
 \begin{itemize}
  \item same as above
 \end{itemize}
 \item {\color{blue} PDG code} \hfill{int}
 \begin{itemize}
  \item identification code of a particle
  \item PDG code is also used as a proxy to nominal values of different physics quantities (like nominal mass, lifetime, ...)
 \end{itemize}
 \item {\color{blue} vector of {\tt StoreArray$\langle$Particle$\rangle$} indices of daughter particles} \hfill{vector$\langle$unsigned$\rangle$}
 \begin{itemize}
  \item As it will be discussed later all \particle objects are stored inside the same StoreArray$\langle$Particle$\rangle$. This vector
  thus holds indices of daughter particles within this StoreArray$\langle$Particle$\rangle$.
 \end{itemize}
 \item {\color{blue} mdst index of an object from which the FS particle is created} \hfill{unsigned}
 \begin{itemize}
  \item only filled for (pre-reconstructed) particles
  \item needed in order to check for overlaps when performing combinations
 \end{itemize}
 \item {\color{blue} type of the \mdst object from which the particle is created } \hfill{int}
 \begin{itemize}
  \item see {\tt EParticleType} enum
  \item motivation is the same as above
 \end{itemize}
 \item {\color{blue}flavor type} \hfill{unsigned}
 \begin{itemize}
  \item in order to support automatic reconstruction of charged conjugated decays
 \end{itemize}
 \item {\color{blue}decay mode identifier} \hfill{unsigned}
 \begin{itemize}
  \item user specified code that uniquely identifies the decay mode in which this particle is reconstructed
 \end{itemize}
\end{itemize}

\paragraph{Transient members}

\begin{itemize}
 \item {\color{blue} pointer to the {\tt StoreArray$\langle$Particle$\rangle$}}\hfill{\tt TClonesArray$\ast$}
 \begin{itemize}
  \item  {\tt StoreArray$\langle$Particle$\rangle$} in which this \particle object is saved
 \end{itemize}
\end{itemize}

\subsubsection{Enums}

\begin{itemize}
 \item {\color{blue}\tt EParticleType}
 \begin{itemize}
  \item {\tt\color{red} c\_Undefined}: \particle is created by user from 4-momentum; has no daughters or BASF2 relation to any \mdst dataobject
  \item {\tt\color{red} c\_Track}: \particle is created from reconstructed \track \mdst dataobject; has BASF2 relation to \track
  \item {\tt\color{red} c\_ECLGamma}: \particle is created from reconstructed {\tt ECLGamma} \mdst dataobject; has BASF2 relation to {\tt ECLGamma}
  \item {\tt\color{red} c\_Pi0}: \particle is created from reconstructed {\tt ECLPi0} \mdst dataobject; has BASF2 relation to {\tt ECLPi0}; internally stores indices of two daughter \particle
  \item {\tt\color{red} c\_KLong}: \particle is created from reconstructed {\tt ECLPi0} \mdst dataobject
  \item {\tt\color{red} c\_MCParticle}: \particle is created from {\tt MCParticle} \mdst dataobject;
  \item {\tt\color{red} c\_Composite}: \particle is reconstructed by the user; internally stores indices to daughter \particle
 \end{itemize}
\end{itemize}

\subsubsection{Constructors}

\paragraph{Constructors from \mdst dataobjects}
\begin{itemize}
 \item the \particle constructor from the following \mdst data objects are provided: {\tt \track, ECLGamma, ECLPi0, MDSTVee, MDSTKlong(?)} and {\tt MCParticle}
 \item the \particle data members are initialized/set within these constructors to the values given in the table below
\end{itemize}
\begin{center}
 \includegraphics[width=0.9\textwidth]{DataModel/figs/particleConstructorsFromMDST.png}
\end{center}

\paragraph{Other constructors}
\begin{itemize}
 \item {\bluett Particle(TLorentzVector momentum, int pdgCode)}
 \begin{itemize}
  \item the simplest constructor
  \item all other members are set to their default values (0)
  \item {\tt particleType} member is set to {\tt c\_Undefined}
 \end{itemize}
 \item {\footnotesize\bluett Particle(TLorentzVector mom, int pdgCode, unsigned flavorType, vector$\langle$unsigned$\rangle$ daugIndices)}
 \begin{itemize}
  \item constructor for composite particles
  \item {\tt particleType} member is set to {\tt c\_Composite}
 \end{itemize}
\end{itemize}


\subsubsection{Member functions}

\paragraph{Getters/Setters for private members}

The table given below provides list of available getters and setters for private members:
\begin{center}
 \includegraphics[width=0.9\textwidth]{DataModel/figs/particleGetterSetter.png}
\end{center}
The getters for kinematic quantities return momentum, position and error matrix as {\tt TLorentzVector, TVector3}, and {\tt TMatrixFSym}, 
respectively. In addition, a function that returns the position (or decay vertex) as the \vertex object is provided.

Special care needs to be taken when it comes to setters for these kinematic quantities, since it can lead to inconsistent 
state, if they are allowed to be modified individually. The momentum and position (decay vertex)
can be correlated and can be thus modified only at once and should not be allowed to be modified individually. 
This means that are no standard setters for these three quantities, like {\tt setMomentum(TLorentzVector), setVertex(TVector3)}, and 
{\tt setErrorMatrix(TMatrixFSym)}, but instead only single public function is provided which enables updating of kinematic quantities
and $\chi^2$ and number of degrees of freedom members:
\begin{itemize}
 \item {\bluett updateMomentum(TLorentzVector, TVector3, TMatrixFSym, float, unsiged)}
\end{itemize}

\paragraph{Additional accessors to other frequently used kinematic quantities}

\begin{itemize}
  \item {\bluett float getP()/getMomentumMagnitude()} -- magnitude of momentum in LAB
  \item {\bluett float getPx()/getPy()/getPz()} -- $x/y/z$ component of momentum in LAB
  \item {\bluett float getMass()} -- invariant mass
  \item {\bluett float getMassError()} -- estimated invariant mass uncertainty
  \item {\bluett float getX()/getY()/getZ()} -- $x/y/z$ component of position (decay vertex)
  \item {\bluett TMatrixFSym getMomentumErrorMatrix()}  -- $4\times4$ momentum error submatrix
  \item {\bluett TMatrixFSym getVertexErrorMatrix()}  -- $3\times3$ momentum error sub-matrix
  \item {\bluett float getMassBeforeKinematicFit()} -- returns the invariant mass before any kinematic fit is performed (with or without mass constraint)
  \begin{itemize}
   \item a transient transient data member can be added (set in the constructor for composite \particle)
   \item or (re-)calculated on demand from original 4-momenta of FS particles (which are never overwritten)
   \item {\bf this should be utility function, since it is needed only during writing out to Ntuple}
  \end{itemize}
  \item {\bluett float getCharge()} -- returns the reconstructed charge 
  \begin{itemize}
   \item determined by summing over the charges of daughter particles
   \item if \particle has no daughter particles then the nominal charge based on its PDG code is returned
  \end{itemize}
\end{itemize}

\paragraph{Additional accessors to nominal quantities} All functions that return nominal values have to start with {\tt getPDG...()} These 
quantities are accessed via {\tt TDatabasePDG}.
\begin{itemize}
 \item {\bluett getPDGCode()} -- returns PDG code
 \item {\bluett getPDGMass()} -- returns nominal mass
 \item {\bluett getPDGWidth()} -- returns intrinsic width
 \item {\bluett getPDGLifetime()} -- returns lifetime
 \item {\bluett getPDGCharge()} -- returns nominal charge
\end{itemize}

\paragraph{Accessors/modifiers of daughters}

\begin{itemize}
 \item {\bluett appendDaughter(Particle)} -- appends (index) of provided \particle to the vector of daughter indices (sets {\tt ParticleType} to {\tt c\_Composite})
 \item {\bluett appendDaughter(int)} -- appends provided index to the vector of daughter indices (sets {\tt ParticleType} to {\tt c\_Composite})
 \item {\bluett unsigned getNDaughters()} -- returns the number of daughters
 \item {\bluett Particle* getDaughter(int)} -- returns the pointer to i-th daughter
 \item {\bluett vector$\langle$Particle*$\rangle$ getDaughters()} -- returns the vector of pointers to all daughters
 \item {\bluett vector$\langle$int$\rangle$ getDaughterIndices()} -- returns the vector of daughter indices
 \item {\bluett vector$\langle$Particle*$\rangle$ getFinalStateDaughters()} -- returns the vector of pointers to FS particle daughters
\end{itemize}

In addition the function that checks whether or not {\tt this} \particle shares any of the FS particles with \particle given as an argument:
\begin{itemize}
 \item {\bluett bool overlapsWith(Particle*)}
\end{itemize}

\newpage
\subsection{\pList}
\secwriter{M. Stari\v c}
\input{DataModel/B2AS-MCParticle}
\subsection{\vertex}
\secwriter{A. Zupanc}
\subsection{\pidLikelihood}
\secwriter{M. Stari\v c}

\newpage
\section{General purpuse analysis modules}

General guidlines:
\begin{itemize}
 \item analysis module performs single well defined analysis action. Grouping of several analysis actions within one 
 module does not follow modular programming paradigm and should be avoided.
 \item clearly state what is the required input and what is the output of a module
 \item clearly state what kind of impact has on existing objects in data store
 \item clearly define the module parameters, define their default and specify valid (range of) values
\end{itemize}


\newpage
\subsection{\pLoader}
\secwriter{M. Stari\v c}

\subsection{\pSelector}
\secwriter{M. Stari\v c}


\subsection{\pCombiner}
\secwriter{M. Stari\v c}


\newpage
\subsection{\pVertexFitter}
\secwriter{A. Zupanc}


\paragraph{General overview}

Module performs a single kinematic fit per call. For example, in the following decay chain of $B$ mesons,
\begin{eqnarray*}
 B^- & \to & D^0 \pi^-\\ 
& & \hookrightarrow K^-\pi^+,
\end{eqnarray*}
two decay vertices exist -- $B$ and $D$ meson decay vertices. Using the \pVertexFitter module both
decay vertices can be reconstructed one after another by performing the following two steps:
\begin{itemize}
 \item 1. step: $D^0 \to K^- \pi^+$ decay vertex reconstruction,
 \item 2. step: $B^- \to D^0 \pi^-$ decay vertex reconstruction (results from step 1 are used and needed).
\end{itemize}

Module {\tt DecayChainVertexFitter} described in section~\ref{sec:modules:decayChainVertexFitter} can fit the entire decay 
chain (both decay vertices) given above in one single step.

\subsubsection{Input and control parameters}

Parameters of the \pVertexFitter module are:
\begin{itemize}
 \item {\bluett pListName} (string): name of the input \pList 
 \item {\bluett decayString} (string): specifies which daughter particles are included in the kinematic fit (see 
 section \ref{sec:tools:decayDescriptor} for a description of the \decayString)
 \begin{itemize}
  \item default value: {\color{red}\tt \bf empty string} $\rightarrow$ all daughter particles are included in the fit
  \item example 1: {\color{red}\tt D0 -> \string^K- \string^pi+}
  \begin{itemize}
   \item all daughters are included 
   \item same as default
  \end{itemize}
  \item example 2: {\color{red}\tt D0 -> \string^pi- \string^pi+ K\_S0}
  \begin{itemize}
   \item $D^0$ decay vertex is determined using $\pi^+$ and $\pi^-$ tracks only; neutral kaon is not included into the fit;
  \end{itemize}
  \item example 3: {\color{red}\tt  D\_s+ -> (phi -> \string^K+ \string^K-) \string^pi+}
  \begin{itemize}
   \item $D_s^+$ decay vertex is determined by fitting $K^+$, $K^-$ and $\pi^+$ tracks to a common point
   \item $\phi$ is strongly decaying resonance with very short lifetime and it makes no sense to fit its decay vertex separately
  \end{itemize}
  \item example 4: {\color{red}\tt  B0 -> (D*+ -> \string^D0 pi+) (D*- -> \string^anti-D0 pi-)}
  \begin{itemize}
   \item $B^0$ decay vertex is determined by fitting $D^0$ and $\overline{D}{}^0$ from $D^{\ast}$ decays to a common point; slow pions 
         are not included;
  \end{itemize}
 \end{itemize}
 \item {\bluett fitType} (string): type of the kinematic fit (see \cite{fitters} for details)
 \begin{itemize}
  \item {\redtt vertex}: this is default value
  \item {\redtt mass}: mass constrained fit (no vertex is determined in the fit)
  \item {\redtt massvertex}: mass-constrained-vertex fit
 \end{itemize}
 \item {\bluett withConstraint} (string): whether an additional constraint on vertex should be imposed
 \begin{itemize}
  \item default value: {\color{red}\tt \bf empty string} $\rightarrow$ no additional constraint
  \item {\redtt ipprofile}: the vertex is constrained to lie within {\tt IPProfile}
  \item {\redtt iptube}: the vertex is constrained to lie within {\tt IPTube}
  \item {\it How could a decay/production vertex of some particle within the decay chain be specified here? Need to discuss it with C. Oswald 
  if \decayString can be used here.}
 \end{itemize}
 \item {\bluett updateDaughters} (bool): whether the momenta of daughter particles should be updated
 \begin{itemize}
  \item default value: {\color{red}\tt \bf false} $\rightarrow$ only mother momentum is updated
  \item {\color{red}\tt \bf true}: mother and daughter momenta are updated (see below for implications/consequences)
 \end{itemize}  
\end{itemize}

\subsubsection{Action description}

Module loops over all mother particles within the specified \pList (via {\tt pListName} module parameter) and performs on each of them specified
kinematic fit as instructed by the user via {\tt decayString, fitType} and {\tt withConstraint} parameters. By default as a result of this
action ({\tt updateDaughters=false}) the following data members of the mother particle are updated with the results of the kinematic fit:
\begin{itemize}
 \item 4-momentum,
 \item position/decay vertex,
 \item $7\times7$ error matrix,
 \item $\chi^2$ value,
 \item number of degrees of freedom.
\end{itemize}
The data members of daughter \particle objects are not changed in any way. By default the module has no impact 
on the \storearray{{\tt Particle}} or \storeobjptr{ParticleList} as illustrated in the figure below (The green font color  for $D^0$ \particle objects with indices 8 and 9
in the {\it after} figure just indicate that the mother \particle objects were modified by the action.).

\newpage
\begin{center}
 {\color{red} Default action ({\tt updateDaughters=false}):}\\
{\color{darkgreen} only mother particle is updated}\\
\vspace{0.2cm} 
{\bf\large Before}\\
\vspace{0.2cm}
\includegraphics[width=0.9\textwidth]{AnalysisModules/figs/vertexingDataStore.pdf}\\
\vspace{0.2cm}
{\bf\large After}\\
\vspace{0.2cm}
\includegraphics[width=0.9\textwidth]{AnalysisModules/figs/vertexingDataStoreB.pdf}
\end{center}

Since in the above case $D^0$ candidates (\particle objects under index 8 and 9) share the same daughter particle (the $K^-$) it is clear
that if we wish to update the daughter momenta we need to make a copies of daughter \particle objects that will be unique to each $D^0$
candidate. Therefore, if it is necessary in the physics analysis to know the fitted momenta of daughter particles (note that this is
in most analyses not the case), user has to turn on the {\tt updateDaughters} flag (set it to {\tt true}). As a consequence the entire 
decay chain (mother and daughter \particle objects) is copied within the \storearray{{\tt Particle}} and the corresponding indices in the
\storeobjptr{ParticleList} are updated as illustrated in the figure below. In addition, all \basf\ \relation{s} are copied.

\newpage
\begin{center}
 {\color{red} Optional action ({\tt updateDaughters=true}):}\\
{\color{darkgreen} mother and daughter particle objects are copied and their momenta are updated}\\
\vspace{0.2cm} 
{\bf\large Before}\\
\vspace{0.2cm}
\includegraphics[width=0.9\textwidth]{AnalysisModules/figs/vertexingDataStore.pdf}\\
\vspace{0.2cm}
{\bf\large After}\\
\vspace{0.2cm}
\includegraphics[width=0.9\textwidth]{AnalysisModules/figs/vertexingDataStoreC.pdf}
\end{center}

\subsection{\ntMaker}
\secwriter{C. Oswald}



\section{Analysis specific analysis modules}

\section{Utility modules}
\subsection{\plMerger}
\secwriter{M. Stari\v c}


\section{Utility functions}

\section{Utility tools}
\subsection{Ntuple tools}
\secwriter{C. Oswald}



% BIBLIOGRAPHY
\begin{thebibliography}{99}

\bibitem{BASF2:manual}
\url{https://belle2.cc.kek.jp/~twiki/bin/view/Computing/Basf2manual}

\bibitem{basf2:coding:convention}
\url{https://belle2.cc.kek.jp/~twiki/bin/view/Computing/CodingConventions}
 
\bibitem{fitters}
Need reference to KFit and Rave.

\end{thebibliography}


\end{document}

