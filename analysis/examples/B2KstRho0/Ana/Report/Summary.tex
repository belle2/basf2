\documentclass[12pt]{article}

\usepackage{amsmath}    % need for subequations
\usepackage{graphicx}   % need for figures
\usepackage{verbatim}   % useful for program listings
\usepackage{color}      % use if color is used in text
\usepackage{subfigure}  % use for side-by-side figures
\usepackage{hyperref}   % use for hypertext links, including those to external documents and URLs

% don't need the following. simply use defaults
\setlength{\baselineskip}{16.0pt}    % 16 pt usual spacing between lines

\setlength{\parskip}{3pt plus 2pt}
\setlength{\parindent}{20pt}
\setlength{\oddsidemargin}{0.5cm}
\setlength{\evensidemargin}{0.5cm}
\setlength{\marginparsep}{0.75cm}
\setlength{\marginparwidth}{2.5cm}
\setlength{\marginparpush}{1.0cm}
\setlength{\textwidth}{150mm}

\begin{comment}
\pagestyle{empty} % use if page numbers not wanted
\end{comment}

% above is the preamble

\begin{document}

\begin{center}
{\large Validation study} \\ % \\ = new line
\clearpage
\end{center}
\section{Introduction}
This document contains results of validation study made for samples of simulated $B^+\to\rho^0K^{*+}(\to K^+\pi^0)$\ decays with Longitudinal and Transverse polarisations. 
For this study events from the simulated samples were reconstructed and underwent some loose selection summarized in Table~\ref{tab:Selection}.
Numbers of generated, reconstructed and selected events, as well as essiciency of these steps together with fraction of self cross-feed canddiates for both samples is presented in Table~\ref{tab:Summary} with detailed efficienies of different steps of the selection in Table~\ref{tab:EffDetailed}. Some distributions of variables commonly used in such analyses are presented in Section~\ref{sec:Vars}. Resolution of angular analysis variables are presented in Section~\ref{sec:Res} and correlations between those variables are studied in Sections~\ref{sec:GCorr} and ~\ref{sec:RCorr}.
\input{Tab_Selection.tex}
\input{Tab_Eff_Summary.tex}
\input{Tab_Rec_Summary.tex}
\section{Variables of angular analysis}
\label{sec:Vars}
\begin{figure}[p]
\centering
\includegraphics[width=0.5\textwidth]{../Decorations/Plots/Bu_Rho0Kst+_K+pi0_Longitudinal_CutEffect/Merged_deltaE.pdf}\hfil
\includegraphics[width=0.5\textwidth]{../Decorations/Plots/Bu_Rho0Kst+_K+pi0_Longitudinal_CutEffect/Merged_Ksthel.pdf}\\
\includegraphics[width=0.5\textwidth]{../Decorations/Plots/Bu_Rho0Kst+_K+pi0_Longitudinal_CutEffect/Merged_Mbc.pdf}\hfil
\includegraphics[width=0.5\textwidth]{../Decorations/Plots/Bu_Rho0Kst+_K+pi0_Longitudinal_CutEffect/Merged_MKST.pdf}\\
\includegraphics[width=0.5\textwidth]{../Decorations/Plots/Bu_Rho0Kst+_K+pi0_Longitudinal_CutEffect/Merged_Mrho.pdf}\hfil
\includegraphics[width=0.5\textwidth]{../Decorations/Plots/Bu_Rho0Kst+_K+pi0_Longitudinal_CutEffect/Merged_rhohel.pdf}\\
\caption{Distributions of fit variables for angular analysis of $B^+\to\rho^0K^{*+}(\to K^+\pi^0)$\ decays with longitudinal polarisation.}
\end{figure}
\clearpage

\begin{figure}[p]
\centering
\includegraphics[width=0.5\textwidth]{../Decorations/Plots/Bu_Rho0Kst+_K+pi0_Transverse_CutEffect/Merged_deltaE.pdf}\hfil
\includegraphics[width=0.5\textwidth]{../Decorations/Plots/Bu_Rho0Kst+_K+pi0_Transverse_CutEffect/Merged_Ksthel.pdf}\\
\includegraphics[width=0.5\textwidth]{../Decorations/Plots/Bu_Rho0Kst+_K+pi0_Transverse_CutEffect/Merged_Mbc.pdf}\hfil
\includegraphics[width=0.5\textwidth]{../Decorations/Plots/Bu_Rho0Kst+_K+pi0_Transverse_CutEffect/Merged_MKST.pdf}\\
\includegraphics[width=0.5\textwidth]{../Decorations/Plots/Bu_Rho0Kst+_K+pi0_Transverse_CutEffect/Merged_Mrho.pdf}\hfil
\includegraphics[width=0.5\textwidth]{../Decorations/Plots/Bu_Rho0Kst+_K+pi0_Transverse_CutEffect/Merged_rhohel.pdf}\\
\caption{Distributions of fit variables for angular analysis of $B^+\to\rho^0K^{*+}(\to K^+\pi^0)$\ decays with transverse polarisation.}
\end{figure}
\clearpage
\begin{figure}[p]
\centering

\includegraphics[width=0.5\textwidth]{../Decorations/Plots/Bu_Rho0Kst+_K+pi0_CutEffect/Merged_ThrustB.pdf}\hfil
\includegraphics[width=0.5\textwidth]{../Decorations/Plots/Bu_Rho0Kst+_K+pi0_CutEffect/Merged_ThrustO.pdf}\\
\includegraphics[width=0.5\textwidth]{../Decorations/Plots/Bu_Rho0Kst+_K+pi0_CutEffect/Merged_CosTBz.pdf}\hfil
\includegraphics[width=0.5\textwidth]{../Decorations/Plots/Bu_Rho0Kst+_K+pi0_CutEffect/Merged_cosTBTO.pdf}\\
\includegraphics[width=0.5\textwidth]{../Decorations/Plots/Bu_Rho0Kst+_K+pi0_CutEffect/Merged_R2.pdf}\hfil
\includegraphics[width=0.5\textwidth]{../Decorations/Plots/Bu_Rho0Kst+_K+pi0_CutEffect/Merged_B_cc1.pdf}\\
\caption{Distributions of some of continuum suppression variables for angular analysis.}
\end{figure}
\clearpage
\begin{figure}[p]
\centering
\includegraphics[width=0.5\textwidth]{../Decorations/Plots/Bu_Rho0Kst+_K+pi0_CutEffect/Merged_EgMax.pdf}\hfil
\includegraphics[width=0.5\textwidth]{../Decorations/Plots/Bu_Rho0Kst+_K+pi0_CutEffect/Merged_EgMin.pdf}\\
\includegraphics[width=0.5\textwidth]{../Decorations/Plots/Bu_Rho0Kst+_K+pi0_CutEffect/Merged_Thetag1.pdf}\hfil
\includegraphics[width=0.5\textwidth]{../Decorations/Plots/Bu_Rho0Kst+_K+pi0_CutEffect/Merged_Thetag2.pdf}\\
\includegraphics[width=0.5\textwidth]{../Decorations/Plots/Bu_Rho0Kst+_K+pi0_CutEffect/Merged_Mpi0Rec.pdf}\\
\caption{Distributions of $\pi^0$ reconstruction quality variables.}
\end{figure}
\clearpage
\begin{figure}[p]
\centering
\includegraphics[width=0.5\textwidth]{../Decorations/Plots/Bu_Rho0Kst+_K+pi0_CutEffect/Merged_VTX_Rank.pdf}\\
\caption{Distribution of the rank of the quality of B-vertex reconstruction for signal and self cross-feed $B^+\to\rho^0K^{*+}(\to K^+\pi^0)$\ candidates.}
\end{figure}
\clearpage
\section{Resolution study}
\label{sec:Res}
This section summarizes resolution of fit variables used in angular analysis. For each variable here is shown 2D scatter plot where generated value of the variable are shown on X axis and difference between reconstructed and generated values (bias) is shown on Y axis. Moreower, plots include average bias and resolution per bins of generated values. All plots are done for truth-matched events after the reconstruction, no additional cuts applied.
\begin{figure}[p]
\centering
\includegraphics[width=0.5\textwidth]{../Decorations/Plots/Bu_Rho0Kst+_K+pi0_B2KstRho/B_deltaE_corr_Gen_vs_Bias.pdf}\hfil
\includegraphics[width=0.5\textwidth]{../Decorations/Plots/Bu_Rho0Kst+_K+pi0_B2KstRho/B_Mbc_corr_Gen_vs_Bias.pdf}\\
\includegraphics[width=0.5\textwidth]{../Decorations/Plots/Bu_Rho0Kst+_K+pi0_B2KstRho/B_MK_Gen_vs_Bias.pdf}\hfil
\includegraphics[width=0.5\textwidth]{../Decorations/Plots/Bu_Rho0Kst+_K+pi0_B2KstRho/B_MR_Gen_vs_Bias.pdf}\\
\includegraphics[width=0.5\textwidth]{../Decorations/Plots/Bu_Rho0Kst+_K+pi0_B2KstRho/B_helK_Gen_vs_Bias.pdf}\hfill
\includegraphics[width=0.5\textwidth]{../Decorations/Plots/Bu_Rho0Kst+_K+pi0_B2KstRho/B_helR_Gen_vs_Bias.pdf}\\
\caption{Results of the resolution study.}
\end{figure}
\clearpage

\include{Bu_Rho0Kst+_K+pi0_MCB2KstRho}
\include{Bu_Rho0Kst+_K+pi0_B2KstRho}
\end{document}