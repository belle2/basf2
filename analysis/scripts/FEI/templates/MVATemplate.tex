\FloatBarrier
\subsubsection{MVA}
\FloatBarrier
\begin{center}
\begin{longtable}{p{5cm}p{7cm}rr}
  \caption{List of variables used in the training.}\\
    \toprule
    Name & Description & Rank & Importance \\
    \midrule
    ${mvaVariables}
    \bottomrule
\end{longtable}
\end{center}
\FloatBarrier
\begin{figure}[h]
  \centering
  \begin{subfigure}[b]{0.32\textwidth}
    \includegraphics[width=\textwidth]{${mvaROCPlot}}
    \caption{Purity over efficiency plot}
  \end{subfigure}\hfill
  \begin{subfigure}[b]{0.32\textwidth}
    \includegraphics[width=\textwidth]{${mvaOvertrainingPlot}}
    \caption{Overtraining plot}
  \end{subfigure}
  \hfill
  \begin{subfigure}[b]{0.32\textwidth}
    \includegraphics[width=\textwidth]{${mvaDiagPlot}}
    \caption{Diag plot}
  \end{subfigure}
  \caption{TMVA Plots for ${mvaType}/${mvaName} using the following configuration: ${mvaConfig}.
           Used target was variable \emph{${mvaTarget}}. The training used ${mvaNTrainSignal} signal
           and ${mvaNTrainBackground} background events. The test sample contained ${mvaNTestSignal} signal
           and ${mvaNTestBackground} background events.}
\end{figure}
\FloatBarrier
