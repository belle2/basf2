\cprotect\section{Overview: \verb|${channelName}|}
\subsection{PreCut}
\FloatBarrier
\begin{figure}[htb]
  \centering
  \begin{subfigure}[b]{0.42\textwidth}
    \includegraphics[width=\textwidth]{${preCutAllPlot}}
    \caption{All possible combinations}
  \end{subfigure}\hfill
  \begin{subfigure}[b]{0.42\textwidth}
    \includegraphics[width=\textwidth]{${preCutSignalPlot}}
    \caption{Only signal combinations}
  \end{subfigure}\hfill
  \begin{subfigure}[b]{0.42\textwidth}
    \includegraphics[width=\textwidth]{${preCutBackgroundPlot}}
    \caption{Only background combinations}
  \end{subfigure}\hfill
  \begin{subfigure}[b]{0.42\textwidth}
    \includegraphics[width=\textwidth]{${preCutRatioPlot}}
    \caption{Ratio of signal and background histograms}
  \end{subfigure}
  \caption{Variable ${preCutVariable} for combinations of daughter candidates of this channel.
           The PreCut range $$ ${preCutRange} $$ for this channel is marked with vertical lines in the plots above.
           The PreCuts were determined on variable ${preCutVariable} with a required total signal efficiency ${preCutEfficiency}.
           For this channel only the efficiency is ${channelEfficiency} with a purity of ${channelPurity}.}
\end{figure}

\FloatBarrier
\subsection{MVA}
\FloatBarrier

\begin{figure}[htb]
  \centering
  \begin{subfigure}[b]{0.42\textwidth}
    \includegraphics[width=\textwidth]{${mvaROCPlot}}
    \caption{Purity over efficiency plot}
  \end{subfigure}\hfill
  \begin{subfigure}[b]{0.42\textwidth}
    \includegraphics[width=\textwidth]{${mvaOvertrainingPlot}}
    \caption{Overtraining plot}
  \end{subfigure}
  \caption{TMVA Plots for ${mvaType}/${mvaName} using following configuration ${mvaConfig}.
           Used target was variable ${mvaTarget}. The training used ${channelNSignal} signal events
           and ${channelNBackground} events.}
\end{figure}


