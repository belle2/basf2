\cprotect\section{Overview: \verb|${channelName}|}
\FloatBarrier
\subsection{Pre-cut determination}
\begin{figure}[htb]
  \centering
  \begin{subfigure}[b]{0.42\textwidth}
    \includegraphics[width=\textwidth]{${preCutAllPlot}}
    \caption{All possible combinations}
  \end{subfigure}\hfill
  \begin{subfigure}[b]{0.42\textwidth}
    \includegraphics[width=\textwidth]{${preCutSignalPlot}}
    \caption{Only signal combinations}
  \end{subfigure}\hfill
  \begin{subfigure}[b]{0.42\textwidth}
    \includegraphics[width=\textwidth]{${preCutBackgroundPlot}}
    \caption{Only background combinations}
  \end{subfigure}\hfill
  \begin{subfigure}[b]{0.42\textwidth}
    \includegraphics[width=\textwidth]{${preCutRatioPlot}}
    \caption{Ratio of signal and background histograms}
  \end{subfigure}
  \caption{Variable \emph{${preCutVariable}} for combinations of daughter candidates of this channel.
           The PreCut range $$ ${preCutRange} $$ for this channel is marked with vertical lines in the plots above.
           The PreCuts were determined on variable \emph{${preCutVariable}} with a required total signal efficiency ${preCutEfficiency}.
           For this channel only the efficiency is ${channelEfficiency} with a purity of ${channelPurity}, corresponding to
           ${channelNSignal} signal and ${channelNBackground} background events.}

\end{figure}

%MVA sections go here
\input{${mvaTexFile}}
