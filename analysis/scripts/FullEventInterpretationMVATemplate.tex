\FloatBarrier
\subsubsection{MVA}

\begin{table}[htb]
  \centering
  \caption{List of variables used in the training.}
  \begin{tabular}{lp{7cm}}
    \toprule
    Name & Description \\
    \midrule
    ${mvaVariables}
    \bottomrule
  \end{tabular}
\end{table}

\begin{figure}[htb]
  \centering
  \begin{subfigure}[b]{0.42\textwidth}
    \includegraphics[width=\textwidth]{${mvaROCPlot}}
    \caption{Purity over efficiency plot}
  \end{subfigure}\hfill
  \begin{subfigure}[b]{0.42\textwidth}
    \includegraphics[width=\textwidth]{${mvaOvertrainingPlot}}
    \caption{Overtraining plot}
  \end{subfigure}
  \hfill
  \begin{subfigure}[b]{0.42\textwidth}
    \includegraphics[width=\textwidth]{${mvaDiagPlot}}
    \caption{Diag plot}
  \end{subfigure}
  \caption{TMVA Plots for ${mvaType}/${mvaName} using the following configuration: ${mvaConfig}.
           Used target was variable \emph{${mvaTarget}}. The training used ${mvaNTrainSignal} signal
           and ${mvaNTrainBackground} background events. The test sample contained ${mvaNTestSignal} signal
           and ${mvaNTestBackground} background events.}
\end{figure}
\FloatBarrier