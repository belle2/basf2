\documentclass[10pt,a4paper]{article}
\usepackage[latin1]{inputenc}
\usepackage{amsmath}
\usepackage{amsfonts}
\usepackage{amssymb}
\usepackage{graphicx}
\usepackage{subcaption}
\usepackage{lmodern}
\usepackage{cprotect}
\usepackage{placeins}
\usepackage{booktabs} %professional tables
\usepackage[left=2cm,right=2cm,top=2cm,bottom=2cm]{geometry}
\author{Thomas Keck\\
Christian Pulvermacher}
\begin{document}

\setlength{\textfloatsep}{5pt plus 1.0pt minus 1.0pt}
\setlength{\floatsep}{5pt plus 1.0pt minus 1.0pt}
\setlength{\intextsep}{5pt plus 1.0pt minus 1.0pt}

\date{\today}
\cprotect\title{Full Event Interpretation Report: \verb|${particleName}|}

\maketitle

\begin{abstract}
In the reconstruction of \verb|${particleName}| ${NUsedChannels} channels out of ${NChannels} possible channels were used.
      This amounts to ${particleNSignal} signal events and ${particleNBackground} background events in total.
      After specific pre cuts for every channel, a post cut on the signal probability of the particle was applied in the range ${postCutRange}.
\end{abstract}

${channelInputs}

\end{document}
