\documentclass[10pt,a4paper]{article}
\usepackage[latin1]{inputenc}
\usepackage{amsmath}
\usepackage{amsfonts}
\usepackage{amssymb}
\usepackage{graphicx}
\usepackage{subcaption}
\usepackage{lmodern}
\usepackage{cprotect}
\usepackage{placeins}
\usepackage{multicol}
\usepackage{booktabs} %professional tables
\usepackage[left=2cm,right=2cm,top=2cm,bottom=2cm]{geometry}
\usepackage{microtype} %optimises spacing, needs to go after fonts
\usepackage{hyperref} %adds links (also in TOC), should be loaded at the very end
\author{Thomas Keck\\
Christian Pulvermacher}
\begin{document}

%\setlength{\textfloatsep}{5pt plus 1.0pt minus 1.0pt}
%\setlength{\floatsep}{5pt plus 1.0pt minus 1.0pt}
%\setlength{\intextsep}{5pt plus 1.0pt minus 1.0pt}

\date{\today}
\title{Full Event Interpretation Report}

\maketitle

\begin{abstract}
This report describes the decay topology of the full event interpretation algorithm.
It contains performed pre and post cuts, trained multivariate analysis methods
and resulting purities and efficiencies. In ${overallSignalEfficiencyInPercent} \% of the events
the full event interpretation reconstructed a B meson decay correctly.
\end{abstract}

\begin{center}
\includegraphics{random.jpg}
\end{center}
\clearpage

\tableofcontents

\FloatBarrier
\clearpage
\section{Summary}
For each final state particle a multivariate analysis method was trained without any previous cut
on the candidates. Afterwards the signal probability calculated by the method was used to perform
a post cut on the final state particles. This reduces combinatorics in the following stages of the full
event interpretation.
\begin{center}
\begin{table}[h]
\caption{Final state particle efficiency and purity before and after the applied post cut.}
\centering
\begin{tabular}{c|rr|rr}
\toprule
Final state &  \multicolumn{2}{c}{Efficiency in \%}  &  \multicolumn{2}{c}{Purity in \%} \\
particle    &  detector & post cut   &  detector & post cut \\
\midrule
${finalStateParticleEPTable}
\bottomrule
\end{tabular}
\end{table}
\end{center}
\FloatBarrier
For each decay channel of each intermediate particle a multivariate analysis method was trained after performing a fast pre cut
on the candidates. Afterwards the signal probability calculated by the method was used to perform
a post cut on the intermediate particle candidates. This reduces combinatorics in the following stages of the full
event interpretation.
\begin{center}
\begin{table}[h]
\caption{Intermediate particle efficiency and purity before and after the applied pre and post cut.}
\centering
\begin{tabular}{c|rrr|rrr}
\toprule
Intermediate &  \multicolumn{3}{c}{Efficiency in \%}  &  \multicolumn{3}{c}{Purity in \%} \\
particle     &  detector & pre cut & post cut   &  detector & pre cut & post cut \\
\midrule
${combinedParticleEPTable}
\bottomrule
\end{tabular}
\end{table}
\end{center}

${mbcInputs}

\FloatBarrier
\clearpage
\section{Particles configuration}
%\begin{multicols}{2}
\begin{verbatim}
${particleConfigurations}
\end{verbatim}
%\end{multicols}
\FloatBarrier
\clearpage
\section{Final state particles}
${finalStateParticleInputs}

\FloatBarrier
\clearpage
${combinedParticleInputs}

\end{document}
