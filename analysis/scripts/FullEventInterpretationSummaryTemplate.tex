\documentclass[10pt,a4paper]{article}
\usepackage[latin1]{inputenc}
\usepackage{amsmath}
\usepackage{amsfonts}
\usepackage{amssymb}
\usepackage{graphicx}
\usepackage{subcaption}
\usepackage{lmodern}
\usepackage{cprotect}
\usepackage{placeins}
\usepackage{booktabs} %professional tables
\usepackage[left=2cm,right=2cm,top=2cm,bottom=2cm]{geometry}
\author{Thomas Keck\\
Christian Pulvermacher}
\begin{document}

\setlength{\textfloatsep}{5pt plus 1.0pt minus 1.0pt}
\setlength{\floatsep}{5pt plus 1.0pt minus 1.0pt}
\setlength{\intextsep}{5pt plus 1.0pt minus 1.0pt}

\date{\today}
\title{Full Event Interpretation Report}

\maketitle

\begin{abstract}
This report describes the decay topology of the full event interpretation algorithm.
It contains performed pre and post cuts, trained multivariate analysis methods
and resulting purities and efficiencies. In ${overallSignalEfficiencyInPercent} \% of the events
the full event interpretation reconstructed a B meson decay correctly.
\end{abstract}

\clearpage

\tableofcontents


\FloatBarrier
\clearpage
\section{Final state particles}
For each final state particle a multivariate analysis method was trained without any previous cut
on the candidates. Afterwards the signal probability calculated by the method was used to perform
a post cut on the final state particles. This reduces combinatorics in the following stages of the full
event interpretation.
\begin{table}[h]
\caption{Final state particle efficiency and purity before and after the applied post cut.}
\begin{tabular}{p{0.10\textwidth}p{0.18\textwidth}|p{0.12\textwidth}p{0.12\textwidth}|p{0.12\textwidth}p{0.12\textwidth}}
Final state particle & Post Cut on Signal Probability &  Efficiency before post cut & Efficiency after post cut  & Purity before post cut & Purity after post cut \\ \hline
${finalStateParticleEPTable}
\end{tabular}
\end{table}

\FloatBarrier
\clearpage
${finalStateParticleInputs}

\section{Intermediate particles}
For each decay channel of each intermediate particle a multivariate analysis method was trained after performing a fast pre cut
on the candidates. Afterwards the signal probability calculated by the method was used to perform
a post cut on the intermediate particle candidates. This reduces combinatorics in the following stages of the full
event interpretation.
\begin{table}[h]
\caption{Intermediate particle efficiency and purity before and after the applied pre and post cut.}
\begin{tabular}{p{0.10\textwidth}p{0.18\textwidth}|p{0.8\textwidth}p{0.8\textwidth}p{0.8\textwidth}|p{0.8\textwidth}p{0.8\textwidth}p{0.8\textwidth}}
Particle & Post Cut on Signal Probability &  Efficiency before pre cut & Efficiency before post cut & Efficiency after post cut  & Purity before pre cut & Purity before post cut & Purity after post cut \\ \hline
${combinedParticleEPTable}
\end{tabular}
\end{table}
\FloatBarrier
\clearpage

${combinedParticleInputs}

\FloatBarrier
\clearpage
\section{MBC Plots}
${mbcInputs}

\end{document}
