\FloatBarrier
\section{CPU time per channel}

\def\colourlist{{${colourList}}}
\tikzset{nodeStyle/.style={text height=\heightof{A},text depth=\depthof{g}, inner sep = 0pt, node distance = -0.15mm}}
\newcount\colourindex \colourindex=-1
\newcommand{\plotbar}[1]{
\begin{tikzpicture}[start chain=going right, nodes = {font=\sffamily}]
  \global\colourindex=-1
  \foreach \percent/\name in {
    #1
  } {
    \ifx\percent\empty\else               % If \percent is empty, do nothing
      \global\advance\colourindex by 1
      \ifnum${numColoursMinusOne}<\colourindex  %back to first colour if we run out
        \global\colourindex=0
      \fi
      \pgfmathparse{\colourlist[\the\colourindex]} % Get color from cycle list
      \edef\color{\pgfmathresult}         %   and store as \color
      \node[nodeStyle, draw, on chain, fill={\color!40}, minimum width=\percent*1.0, minimum height=12] {\name};
    \fi
  };
\end{tikzpicture}
}

\FloatBarrier
  \begin{longtable}{lrcrr}
  \caption{Total CPU time spent in event() calls for each channel. Bars show ${barLegend}, in this order. Does not include I/O, initalisation, training, post-cuts etc.}\\
    \toprule
    Decay & CPU time & by module & per (true) candidate & Relative time \\
    \midrule
    ${cpuTimeStatistics}
    \bottomrule
  \end{longtable}
\FloatBarrier
